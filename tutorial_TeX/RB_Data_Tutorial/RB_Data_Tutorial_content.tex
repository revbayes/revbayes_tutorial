\section{Overview}


This tutorial describes the data formats that are used in \RevBayes.


\subsection*{Requirements}
We assume that you have read and hopefully completed the following tutorials:
\begin{itemize}
\item RB\_Getting\_Started
\end{itemize}



%
%
%
\newpage
\FloatBarrier
\section{Molecular Sequence Data}

\bigskip
\subsection{Getting Started}


\exs{Download data and output files from: \href{http://revbayes.github.io/tutorials.html}{http://revbayes.github.io/tutorials.html}}
\exs{Open the file \cl{primates\_cytb.nex} in your text editor. 
This file contains the nucleotide sequences of the cytochrome B gene sampled from 13 species (Box 1). 
The elements of the \cl{DATA} block indicate the data type, number of taxa, and sequence length.}


\begin{center}
Box 1: A fragment of the NEXUS file containing the ITS sequences for this exercise. \\
\end{center}
{\tt \scriptsize \begin{framed}
\begin{lstlisting}
#NEXUS 

Begin data;
Dimensions ntax=13 nchar=673;
Format datatype=DNA missing=? gap=-;
Matrix
Trig_excelsa   
TCGAAACCTG...
Fagus_engleriana   
TCGAAACCTG...
Fagus_crenata1   
TCGAAACCTG...
Fagus_japonica2   
TCGAAACCTG...
Fagus_japonica1   
TCGAAACCTG...
Fagus_orientalis   
TCGAAACCTG...
Fagus_sylvatica   
TCGAAACCTG...
Fagus_lucida1   
TCGAAACCTG...
Fagus_lucida2   
TCGAAACCTG...
Fagus_crenata2   
TCGAAACCTG...
Fagus_grandifolia   
TCGAAACCTG...
Fagus_mexicana   
TCGAAACCTG...
Fagus_longipetiolata   
TCGAAACCTG...
	;
End;
\end{lstlisting}
\end{framed}}


\subsection{Loading Molecular Sequence Data}
We can read the data into \RevBayes~ using the \cl{readDiscreteCharacterData()} function. 
%This function returns a \textit{vector} of data matrices and, even though there is only one element in the vector, we must index that element using the \cl{[1]} notation. 
%(You will also note that list indexing in Rev starts with \cl{1} like in the R language.)
{\tt \begin{snugshade*}
\begin{lstlisting}
data <- readDiscreteCharacterData("data/primates_cytb.nex")
\end{lstlisting}
\end{snugshade*}}

\subsection{Querrying Dataset Attributes}
When a dataset has been loaded into \RevBayes, we can query relevant \Rev~variables. 
To report the current value of any variable, simply type the variable name and press enter. 
For example, the \cl{data} variable returns general information about the sequence alignment:
{\tt \begin{snugshade*}
\begin{lstlisting}
data
|*   DNA character matrix with 23 taxa and 1141 characters
|*   =====================================================
|*   Origination:                      primates_cytb.nex
|*   Number of taxa:                   23
|*   Number of included taxa:          23
|*   Number of characters:             1141
|*   Number of included characters: 1141
|*   Datatype:                         DNA
\end{lstlisting}
\end{snugshade*}}

The \cl{data} variable has \textit{member functions} that we can use to retrieve specific attributes of the dataset. 
These member functions include the number of taxa (\cl{data.ntaxa()}), the sequence length (\cl{data.nchar()}), etc.
{\tt \begin{snugshade*}
\begin{lstlisting}
data.ntaxa()
|*   23
\end{lstlisting}
\end{snugshade*}}

Available \textit{member functions} for the data variable are listed in Table \ref{tab:mem_fns}.


\begin{table}[h!]
\centering
\caption{\small Available member functions for the data variable.} \label{tab:mem_fns}
\centering
\begin{tabularx}{\textwidth}{ll}
\toprule
	Function name 	  				& Type							 \\
	\midrule
	\cl{chartype} 	 					& 	 String function () 				\\ 
	 \rowcolor{gray!25}
 	\cl{excludeAll} 	 					& 	 void function () 				\\ 
 	\cl{excludeCharacter}	 			& 	 void function (Natural ) 			\\ 
	\rowcolor{gray!25}
 	\cl{excludeCharacter}	 			& 	 void function (Natural [ ]) 			\\ 
 	\cl{getEmpiricalBaseFrequencies} 	 	& 	 Simplex function ()				\\ 
	\rowcolor{gray!25}
 	\cl{getNumInvariantSites} 	 		& 	 Natural function ()				\\ 
 	\cl{includeAll} 	 					& 	 void function ()				\\ 
	\rowcolor{gray!25}
 	\cl{includeCharacter} 	 			& 	 void function (Natural )			\\ 
 	\cl{includeCharacter} 	 			& 	 void function (Natural [ ] )			 \\ 
	\rowcolor{gray!25}
 	\cl{ishomologous} 	 				& 	 Bool function ()				\\ 
 	\cl{methods} 	 					& 	 void function ()				\\ 
	\rowcolor{gray!25}
 	\cl{names} 	 					& 	 String [ ] function ()				\\ 
 	\cl{nchar} 	 					& 	 Natural function ()				\\ 
	\rowcolor{gray!25}
 	\cl{ntaxa} 	 						& 	 Natural function ()				\\ 
 	\cl{removeTaxa} 	 				& 	 void function (String )			\\ 
	\rowcolor{gray!25}
 	\cl{removeTaxa} 	 				& 	 void function (String [ ] )			\\ 
 	\cl{setCodonPartition} 	 			& 	 void function (Natural )			\\ 
	\rowcolor{gray!25}
 	\cl{setCodonPartition} 	 			& 	 void function (Natural [ ] )			\\ 
 	\cl{setNumStatesPartition} 	 		& 	 void function (Natural )			\\ 
	\rowcolor{gray!25}
 	\cl{setTaxonName} 	 				& 	 void function (String current, String new)	\\ 
 	\cl{show} 	 						& 	 void function ()				\\ 
	\rowcolor{gray!25}
 	\cl{size} 	 						& 	 Natural function ()				\\ 
	\bottomrule
\end{tabularx}
\end{table}


%{\tt \begin{snugshade*}
%\begin{lstlisting}
%data.names()	
%|*    [ "Saimiri_sciureus", "Callicebus_donacophilus", "Cebus_albifrons", ...]
%\end{lstlisting}
%\end{snugshade*}}
%

%BRM: I think we may want to provide a (semi) comprehensive list of member functions/attributes here, at least common ones, perhaps in a table?
%If so, can you tell me where to find them??

\subsection{Concatenating Sequences}
We can combine two or more datasets using the \cl{concatenate} function.
First, we will read in two datasets; the first is an alignment of primate cytb sequences, the second is an alignment of cox2 sequences:

{\tt \begin{snugshade*}
\begin{lstlisting}
data_cytb <- readDiscreteCharacterData("data/primates_cytb.nex")
data_cox2 <- readDiscreteCharacterData("data/primates_cox2.nex")
\end{lstlisting}
\end{snugshade*}}

Next, we will concatenate these two alignments using the \cl{concatenate} function. 
This returns a single data matrix that combines the sequences of both gene regions.

{\tt \begin{snugshade*}
\begin{lstlisting}
data <- concatenate(data_cytb, data_cox2)
\end{lstlisting}
\end{snugshade*}}

We can confirm this by querying the \cl{data} variable: 

{\tt \begin{snugshade*}
\begin{lstlisting}
data
|*      DNA character matrix with 23 taxa and 1852 characters
|* =====================================================
|* Origination:                   primates_cytb.nex
|* Number of taxa:                23
|* Number of included taxa:       23
|* Number of characters:          1852
|* Number of included characters: 1852
|* Datatype:                      DNA
\end{lstlisting}
\end{snugshade*}}
%BRM: note that the ``origination'' lists the name of the first gene in the concatenated alignment


\subsection{Excluding/Including Taxa}
We can exclude species from an alignment that is currently in memory using the \cl{removeTaxa} function.
For example, we could exclude the outgroup species \textit{Saimiri sciureus} from our concatenated primate alignment (\cl{data}) by typing:

{\tt \begin{snugshade*}
\begin{lstlisting}
data.removeTaxa("Saimiri_sciureus")
\end{lstlisting}
\end{snugshade*}}

We can then confirm the removal of a species by checking the number of remaining taxa:
{\tt \begin{snugshade*}
\begin{lstlisting}
data.ntaxa()
|*   22
\end{lstlisting}
\end{snugshade*}}
The number of species has decreased by one, as expected.
We can confirm that we have excluded \textit{Saimiri sciureus} by typing:
{\tt \begin{snugshade*}
\begin{lstlisting}
data.names()	
|*    [ "Callicebus_donacophilus", "Cebus_albifrons", "Alouatta_palliata", ...]
\end{lstlisting}
\end{snugshade*}}

\subsection{Excluding/Including Sites or Genes}
We can exclude a single site (or set of sites) from an alignment that is currently in memory using the \cl{excludeCharacter} function.
For example, we could exclude the first site in our concatenated primate alignment (\cl{data}) by typing:

{\tt \begin{snugshade*}
\begin{lstlisting}
excludeCharacter([1])
\end{lstlisting}
\end{snugshade*}}
[Note that sites of an alignment are indexed from 1 to $N$.]
We can confirm the removal of a site by checking the number of remaining sites:

{\tt \begin{snugshade*}
\begin{lstlisting}
data.nchar()
|*   1851
\end{lstlisting}
\end{snugshade*}}
The number of sites has decreased by one, as expected.
We can return the excluded site to our alignment using the \cl{includeCharacter} function:

{\tt \begin{snugshade*}
\begin{lstlisting}
includeCharacter([1])
\end{lstlisting}
\end{snugshade*}}

%We can then confirm the inclusion of the excluded site by checking the number of alignment length:
%{\tt \begin{snugshade*}
%\begin{lstlisting}
%data.nchar()
%|*   1852
%\end{lstlisting}
%\end{snugshade*}}
%The number of sites has increased by one, as expected.

We can similarly exclude/include a range of sites, {\it e.g.}, corresponding to a gene region.
Here, we will exclude all $1141$ sites comprising the cytb gene region from our concatenated alignment:
{\tt \begin{snugshade*}
\begin{lstlisting}
data.excludeCharacter(1:1141)
\end{lstlisting}
\end{snugshade*}}

We can check the number of remaining sites, which comprise the cox2 gene region:
{\tt \begin{snugshade*}
\begin{lstlisting}
data.nchar()
|*   711
\end{lstlisting}
\end{snugshade*}}

We can easily return the excluded cytb sequences by typing:
{\tt \begin{snugshade*}
\begin{lstlisting}
data.includeCharacter(1:1141)
\end{lstlisting}
\end{snugshade*}}

It is also possible to exclude/include all sites using the \cl{excludeAll} and \cl{includeAll} commands.



%\section{Discrete-Trait Data}
%
%
%
%\section{Continuous-Trait Data}
%

\section{Biogeographical Data}

For concreteness, this section focuses on the Hawaiian {\it Psychotria} dataset used in \citet{ree08}.

\subsection{Nexus file}

The data file contains a matrix of binary characters corresponding to the observed ranges of the study taxa.

\begin{framed}
\begin{lstlisting}
#NEXUS

begin data;
  dimensions ntax=19 nchar=4;
  format datatype=standard symbols = "01";
  matrix
    P_mariniana_Kokee2  1000
    P_mariniana_Oahu    0100
    ...
    P_hexandra_Oahu     0100
  ;
end;
\end{lstlisting}
\end{framed}

Geographic range data is stored in standard Nexus format.
In the {\tt data} block, the first line gives the dimensions of the data matrix and the second line indicates we will be using binary characters.
The four characters correspond to areas defined by the geography file (next subsection).
Rows in the {\tt matrix} block correspond to taxa and their geographic range data, while columns give in which areas each taxon is present (1) or absent (0).
For example, {\tt P\_hexandra\_Oahu} is endemic to area 2.

\subsection{Atlas file}

The atlas file describes the biogeographic state space as it might change over time.

\begin{framed}
\begin{lstlisting}[style=textboxSmall]
{
  "name":"HawaiianArchipelago5my",
  "epochs": [
  {
    "name":"epoch1",
    "start_age":100.0,
    "end_age":3.7,
    "areas":
      [{ "name":"K","latitude":19.6,"longitude":-155.5,"dispersalValues":[1,0,0,0]},
       { "name":"O","latitude":19.6,"longitude":-155.5,"dispersalValues":[0,0,0,0]},
       { "name":"M","latitude":19.6,"longitude":-155.5,"dispersalValues":[0,0,0,0]},
       { "name":"H","latitude":19.6,"longitude":-155.5,"dispersalValues":[0,0,0,0]}]
  },
  {
    "name":"epoch2",
    ...
  },
  {
    "name":"epoch3",
    ...
  },
  {
    "name":"epoch4",
    "start_age":0.5,
    "end_age":0.0,
    "areas":
      [{"name":"K","latitude":22.1,"longitude":-159.5,"dispersalValues":[1,1,1,1]},
       {"name":"O","latitude":21.5,"longitude":-158.0,"dispersalValues":[1,1,1,1]},
       {"name":"M","latitude":20.8,"longitude":-156.3,"dispersalValues":[1,1,1,1]},
       {"name":"H","latitude":19.6,"longitude":-155.5,"dispersalValues":[1,1,1,1]}]
    }
  ]
}
\end{lstlisting}
\end{framed}

The atlas stores information in JSON (JavaScript Object Notation) format.
JSON is a lightweight format used to assign values to variables in a hierarchical manner.
There are three main tiers to the hierarchy in the Atlas file: the atlas, the epoch, and the area.
In the lowest tier, each area corresponds to a character in the model and is assigned it's own properties.
In the middle tier, each epoch contains the set of homologous areas (characters) that may be part of a species' range, but importantly the properties of these areas may take on different values during different intervals of time, as given by the {\tt start\_age} and {\tt end\_age} variables.
Because the tree and range evolution model also operate on units of geological time, the rates of area gain and loss can condition on areas' properties as a function of time.
Sometimes these models are called stratified models or epochal models.
Finally, the atlas contains the array of epochs in the highest tier.

Each area is assigned a {\tt latitude} and {\tt longitude} to represent its geographical coordinates, ideally the centroid of the area.
If a centroid does not represent the distance between areas, splitting the area into multiple smaller areas is reasonable.
Here, the {\tt latitude} and {\tt longitude} change in each of the four epochs, where they begin at the current location of Hawaii and drift northwesterly until they reach their current positions.

In addition, each area is marked as habitable or not using the {\tt dispersalValues} array.
The elements in the array correspond to the other areas defined in the analysis.
For example, in {\tt epoch1}, Kauai's {\tt dispersalValues} is equal to {\tt [ 1,0,0,0 ]}, which indicates Kauai exists at that point in time but it is not in contact with any other areas, i.e. the range in that area cannot expand into other areas.
The {\tt dispersalValues} for Oahu, Maui, and Hawaii are all equal to {\tt [ 0,0,0,0 ]}, meaning no species may be present in that area during the time interval of {\t epoch1} during ages from 10.0 to 3.7. In contrast, {\tt epoch4}, from ages 0.5 to the present, range expansions may occur between any pair of areas and any area may be included in a species' range.


\vspace{5cm}
Questions about this tutorial can be directed to: \\\vspace{-10mm}
\begin{itemize}
\item Tracy Heath (email: \href{mailto:tracyh@berkeley.edu}{tracyh@berkeley.edu}) \\\vspace{-8mm}
\item Michael Landis (email: \href{mailto:mlandis@berkeley.edu}{mlandis@berkeley.edu}) \\\vspace{-8mm} 
\item Sebastian H\"{o}hna (email: \href{mailto:sebastian.hoehna@gmail.com}{sebastian.hoehna@gmail.com}) \\\vspace{-8mm}
\item Brian R. Moore (email: \href{mailto:brianmoore@ucdavis.edu}{brianmoore@ucdavis.edu}) \\\vspace{-8mm}
\end{itemize}


%\bibliographystyle{mbe}
%\bibliography{bib_tex/master_refs}
