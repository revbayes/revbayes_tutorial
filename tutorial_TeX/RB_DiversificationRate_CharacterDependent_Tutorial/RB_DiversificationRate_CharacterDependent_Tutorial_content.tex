\section{Estimating Character-Dependent Speciation \& Extinction Rates}

\subsection{Outline}

This tutorial describes how to specify a character-dependent branching-process models in \RevBayes;
a birth-death process where diversification rates are dependent on the state of a discrete character \citep{Maddison2007,Fitzjohn2012,Beaulieu2016}.
The probabilistic graphical model is given for this tutorial.
Finally, you will estimate character-dependent speciation and extinction rates using Markov chain Monte Carlo (MCMC).


\subsection{Requirements}
We assume that you have read and hopefully completed the following tutorials:
\begin{itemize}
\item \href{https://github.com/revbayes/revbayes_tutorial/raw/master/tutorial_TeX/RB_Getting_Started/RB_Getting_Started.pdf}{Getting started}
\item \href{https://github.com/revbayes/revbayes_tutorial/raw/master/tutorial_TeX/RB_Basics_Tutorial/RB_Basics_Tutorial.pdf}{\Rev basics}
\item \href{https://github.com/revbayes/revbayes_tutorial/raw/master/tutorial_TeX/RB_DiversificationRate_Tutorial/RB_DiversificationRate_Tutorial.pdf}{Basic Diversification Rate Estimation}
\end{itemize}
Note that the \href{https://github.com/revbayes/revbayes_tutorial/raw/master/tutorial_TeX/RB_Basics_Tutorial/RB_Basics_Tutorial.pdf}{\Rev basics tutorial} introduces the basic syntax of \Rev but does not cover any phylogenetic models.
You may skip the \href{https://github.com/revbayes/revbayes_tutorial/raw/master/tutorial_TeX/RB_Basics_Tutorial/RB_Basics_Tutorial.pdf}{\Rev basics tutorial} if you have some familiarity with \R.
We tried to keep this tutorial very basic and introduce all the language concepts and theory on the way.
You may only need the \href{https://github.com/revbayes/revbayes_tutorial/raw/master/tutorial_TeX/RB_Basics_Tutorial/RB_Basics_Tutorial.pdf}{\Rev basics tutorial} for a more in-depth discussion of concepts in \Rev.


%%%%%%%%%%%%%%
%%   Data   %%
%%%%%%%%%%%%%%
\section{Data and files}

We provide the data file(s) which we will use in this tutorial.
You may want to use your own data instead.
In the \cl{data} folder, you will find the following files
\begin{itemize}
\item \href{https://github.com/revbayes/revbayes_tutorial/raw/master/RB_DiversificationRate_CharacterDependent_Tutorial/data/primates\_tree.nex}{primates\_tree.nex}: Dated primates phylogeny including 233 out of 367 species from \cite{MagnusonFord2012}.
\item \href{https://github.com/revbayes/revbayes_tutorial/raw/master/RB_DiversificationRate_CharacterDependent_Tutorial/data/primates_habitat.nex}{primates\_habitat.nex}: Habitat data from \cite{MagnusonFord2012}. There is also a larger set of \href{https://github.com/revbayes/revbayes_tutorial/raw/master/RB_DiversificationRate_CharacterDependent_Tutorial/data/primates_morph.nex}{discrete morphological characters}. The type of characters are described in the file \href{https://github.com/revbayes/revbayes_tutorial/raw/master/RB_DiversificationRate_CharacterDependent_Tutorial/data/primates_morph_description.txt}{primates\_morph\_description.txt}.
\end{itemize}


\impmark{Open the tree \cl{data/primates\_tree.nex} in FigTree.}

\impmark{Open the character data file \cl{data/primates\_morph.nex} in a text editor.}


\bigskip
\section{The Theory behind Character-dependent diversification rate models}\label{sec:BiSSE_Theory}

\RevBayes implements the multi-state extension of \BiSSE, just as implemented in \diversitree. 
Here we will first describe the general theory about the model, borrowing heavily from the supplementary material of \cite{Moore2016}.
Then we will show how to run this analysis in \RevBayes.

%%%%%%%%%%%%%%%%%%%%%%%%%%%%
%%%    LIKELIHOOD  FN    %%%
%%%%%%%%%%%%%%%%%%%%%%%%%%%%
\subsection{Likelihood function}\label{section:likelihood}
	We introduce the theory and likelihood function of the \BiSSE (Binary State Speciation and Extinction) model developed in \cite{Maddison2007}.
	Specifically, the \BiSSE model assumes a finite number of processes (two; one for each of the binary states), where the process to which each extant species belongs is known (indeed, it is an observed discrete trait).
	The general approach adopted by \BiSSE models is to derive a set of ordinary differential equations (ODEs) that describe how the probability of observing a descendant clade changes along a branch in the observed phylogeny.
	Each equation in this set describes how the probability of observing the clade changes through time if it is in a particular process over that time period; collectively, these equations are called $\frac{ \mathrm{d}D_{N,i}(t)}{\mathrm{d}t}$, where the subscript $N$ refers to the descendant clade and the subscript $i$ refers to $i^\text{th}$ process.

	Computing the likelihood proceeds by establishing an initial value problem: in principle, if we know the probabilities of observing a lineage at some specific time (\EG the present), and know how those probabilities change over time (described by the ODEs), then we can compute the probabilities of observing those lineages at some earlier time (\EG the root).
	Assuming that there are exactly $k$ possible states, we initialize $k$ probabilities at each tip in the phylogeny; we then compute how each of those $k$ probabilities changes down each branch in the tree using the above set of $k$ ODEs.
	At each node in the tree, we take the product of each of the $k$ probabilities for the descendants of that node (multiplied by the instantaneous speciation rate for each of the $k$ states to account for the observed speciation event at the node) as the initial values for the ancestral branch subtending that node.
	Proceeding in this way down the tree results in a set of $k$ probabilities at the root; these $k$ probabilities represent the probability of observing the phylogeny conditional on the root being in each of the processes (\IE the $i^\text{th}$ conditional probability is the probability of observing the tree given that the root is in process $i$).
	The overall likelihood of the tree is a weighted average of the $k$ probabilities at the root, where the weighting scheme represents the assumed probability that the root was in each of the $k$ states.

	As with all birth-death process models, special care must be taken to account for the possibility of extinction.
	Specifically, the above ODEs must accommodate lineages that may arise along each branch in the tree that subsequently go extinct before the present (and so are unobserved).
	This requires a second set of $k$ ODEs, $\frac{ \mathrm{d}E_{i}(t)}{\mathrm{d}t}$, which define how the probability of extinction in process $i$ changes over time.
	These ODEs must be solved to compute the differential equations $\frac{ \mathrm{d}D_{N,i}(t)}{\mathrm{d}t}$, as we will demonstrate when we derive both sets of equations in the following sections.

	This framework therefore requires four distinct pieces of information to compute the likelihood of the data:
%	\begin{enumerate}
%		\item A set of ordinary differential equations describing how the probability of the data (observed lineages) changes through time, $\frac{ \mathrm{d}D_{N,i}(t)}{\mathrm{d}t}$.
%		\item A set of ordinary differential equations describing how the extinction probability of unobserved (extinct or unsampled) lineages changes through time, $\frac{ \mathrm{d}E_{i}(t)}{\mathrm{d}t}$.
%		\item An appropriate set of initial conditions.
%		\item An appropriate weighting scheme for the root probabilities.
%	\end{enumerate}

	In the following sections we detail how each of these components is determined for increasingly complex birth-death process models.


\subsubsection{Binary-state space of diversification processes}
Consider a (time-independent) birth-death process with two possible states (\EG a binary state), with diversification rates $\{\lambda_0, \mu_0\}$ and $\{\lambda_1, \mu_1\}$.
We define $D_{N,0}(t)$ as the probability of observing lineage $N$ descending from a particular branch at time $t$, given that the lineage at that point is in state 0 (with rate parameters $\lambda_{0}$, and $\mu_{0}$).
To compute the probability of observing the lineage at some earlier point, $ D_{N,0}(t + \Delta t)$, we enumerate all possible events that could occur within the interval $\Delta t$.
Assuming that $\Delta t$ is small---so that the probability of any two events occurring in the interval is negligible---there are four possible scenarios within the interval:
\begin{enumerate}
	\item nothing happens in the interval;
	\item the process changes $0 \rightarrow 1$;
	\item a speciation event occurs and the left descendant subsequently goes extinct before the present, or;
	\item a speciation event occurs and the right descendant subsequently goes extinct before the present.
\end{enumerate}
We can thus compute $D_{N,0}(t + \Delta t)$ as (see \cite{Maddison2007} and \cite{FitzJohn2009} for a more complete elucidation):
\begin{align}
	D_{N,0}(t + \Delta t) = & \;(1 - \mu_0\Delta t) \times & \text{in all cases, no extinction of the observed lineage} \label{equation:ProbDeltaT} \\
			    		   & \;[  (1 - q\Delta t)(1 - \lambda_0\Delta t)D_{N,0}(t) & \text{Case (1) nothing happens} \nonumber\\
						   & \; + q_{01}\Delta t (1 - \lambda_0\Delta t)D_{N,1}(t) & \text{Case (2) process change but no speciation} \nonumber\\
						   & \; + (1 - q_{01}\Delta t)\lambda_0\Delta t E_0(t)D_{N,0}(t) & \text{Case (3) no process change, speciation, extinction} \nonumber\\
						   & \; + (1 - q\Delta t)\lambda_0\Delta t E_0(t)D_{N,0}(t)] & \text{Case (4) no process change, speciation, extinction} \nonumber
\end{align}
A matching equation can be written down for $D_{N,1}(t+\Delta t)$.

Define $E_0(t)$ as the probability that a lineage in state 0 at time $t$ goes extinct before the present.
To determine the extinction probability at an earlier point, $E_0(t+\Delta t)$, we can again enumerate all the possible events in the interval $\Delta t$:
\begin{enumerate}
	\item the lineage goes extinct within the interval;
	\item the lineage neither goes extinct nor speciates, resulting in a single lineage that must eventually go extinct before the present;
	\item the lineage neither goes extinct nor speciates, but there is a state change, resulting in a single lineage that must go extinct before the present, or;
	\item the lineage speciates in the interval, resulting in \emph{two} lineages that must eventually go extinct before the present.
\end{enumerate}

\begin{align}
	E_0(t + \Delta t) = &\; \mu_0\Delta t +	& \text{Case (1) extinction in the interval} \label{equation:ExtDeltaT}\\
				     & (1 - \mu_0\Delta t) \times & \text{no extinction in the interval and \dots} \nonumber\\
				     & \;[(1-q_{01}\Delta t)(1-\lambda_0\Delta t)E_0(t) & \text{Case (2) nothing happens, but subsequent extinction} \nonumber\\
				     & \;+ q_{01}\Delta t (1-\lambda_0\Delta t)E_1(t) & \text{Case (3) process change and subsequent extinction} \nonumber\\
				     & \;+ (1 - q\Delta t) \lambda_0\Delta t E_0(t)^2] & \text{Case (4) speciation and subsequent extinctions} \nonumber
\end{align}
Again, a matching equation $E_1(t+\Delta t)$ can be written down.

\subsection{Extension to a multi-state birth-death process}
We can expand the \BiSSE model to accommodate an arbitrary number of processes, $k$, by writing a set of $k$ difference equations $D_{N,0}(t+\Delta t), D_{N,1}(t+\Delta t), \ldots, D_{N,k}(t+\Delta t)$:
\begin{align}
		D_{N,i}(t + \Delta t) = & \;(1 - \mu_i\Delta t) \times \label{equation:DKStates} \\
		& \;[  (1 - \sum\limits_{j \neq i}^k q_{ij}\Delta t)(1 - \lambda_i\Delta t)D_{N,i}(t) \nonumber\\
		& \; + (1 - \lambda_i\Delta t)  \sum\limits_{j \neq i}^k q_{ij}\Delta t D_{N,j}(t) \nonumber\\
		& \; + 2 (1 - \sum\limits_{j \neq i}^k q\Delta t) \lambda_i\Delta t E_i(t)D_{N,i}(t)] \nonumber
\end{align}
along with $E_0(t+\Delta t), E_1(t+\Delta t), \ldots, E_k(t+\Delta t)$:
\begin{align}
	E_i(t + \Delta t) = &\; \mu_i\Delta t +	\label{equation:ExtKStates}	\\
				    	    & (1 - \mu_i\Delta t) \times \nonumber\\
					    & \;[(1-\sum\limits_{j \neq i}^k q_{ij}\Delta t)(1-\lambda_i\Delta t)E_i(t) \nonumber\\
					    & \;+ (1-\lambda_i\Delta t) \sum\limits_{j \neq i}^k q_{ij}\Delta t E_j(t) \nonumber\\
					    & \;+ (1 - \sum\limits_{j \neq i}^k q\Delta t) \lambda_i\Delta t E_i(t)^2] \nonumber
\end{align}
It is possible to derive differential equations from the difference equations \ref{equation:DKStates} and \ref{equation:ExtKStates} (see \cite{Maddison2007} for the two-process case and \cite{FitzJohn2012} for the $k$-process case).
For the general $k$-process case, the differential equations are:
\begin{align*}
	\frac{\mathrm{d}D_{N,i}(t)}{\mathrm{d}t} = & - \left(\lambda_i + \mu_i + \sum\limits_{j \neq i}^k q\right)D_{N,i}(t) + 2\lambda_iE_i(t)D_{N,i}(t) + \sum\limits_{i \neq j}^k q D_{N,j}(t) \\
   	\frac{\mathrm{d}E_i(t)}{\mathrm{d}t} = & - \left(\lambda_i + \mu_i + \sum\limits_{j \neq i}^k q \right)E_i(t) + \lambda_iE_i(t)^2 + \mu_i + \sum\limits_{i \neq j}^k q E_j(t)
\end{align*}

Initial probabilities are assigned according to the observed discrete states: if species $i$ has state $j$, then $D_{i,j}(0) = 1$ for the observed state, and $D_{i,j}(0) = 0$ for all other ($\neq j$) states.
If the state is not observable, then $D_{i,j}(0) = 1$ for all $j$, since all states have probability 1 of producing the observation; this is analogous to the treatment of missing or ambiguous states in conventional phylogenetic likelihood calculation, \emph{c.f.}, \cite{Felsenstein2004}.
Initial extinction probabilities are set to 0 (since there is no time for extinction to occur at the present) or to $1-\rho_i$ if incomplete taxon sampling is used. 
Root probabilities are either weighted using equal probabilities (uniformly), by a vector of pre-defined root stationary probabilities (informative), or by the stationary distribution of the model, to compute the overall likelihood of the data.


\begin{table}[t!]
	\centering
	\caption{\bf{\BiSSE model parameters and their interpretation}} \label{tab:param}
	\begin{tabular}{ l l l }
		\toprule
		Parameter & Interpretation \\
		\midrule
		$\Psi$ & Phylogenetic tree with divergence times.\\
		\rowcolor{gray!15} $T$ & The root age.\\
%		\rowcolor{gray!15} $\gamma$ & Prior mean of the Poisson rate $\Lambda$.\\
%		$\Lambda$ & Prior mean of the Poisson-distributed number $k$ of shift events.\\
		$q_{01}$ & The rate of shifts from 0 to 1.\\
		\rowcolor{gray!15} $q_{10}$ & The rate of shifts from 1 to 0.\\
		$\lambda_i$ & Speciation rate for the $i^\text{th}$ state.\\
		\rowcolor{gray!15} $\mu_i$ & Extinction rate for the $i^\text{th}$ state.\\
	\end{tabular}
\end{table}


\section{Character-dependent diversification rates}\label{sec:CDBDP}
Now let's start to analyze an example in \RevBayes using the BiSSE model.

\subsection{Read the tree}

Begin by reading in the observed tree and the morphological data. 
We have both stored in separate nexus files.
{\tt \begin{snugshade*}
\begin{lstlisting}
observed_phylogeny <- readTrees("data/primates_tree.nex")[1]
data <- readCharacterData("data/primates_solitariness.nex")
\end{lstlisting}
\end{snugshade*}}
Note, the character-dependent birth-death process currently uses always the first character/site in the alignment file.
We have therefore split the morphological dataset into several small files that include only one morphological character each.

From the tree, we can get some helpful variables:
{\tt \begin{snugshade*}
\begin{lstlisting}
taxa <- observed_phylogeny.taxa()
\end{lstlisting}
\end{snugshade*}}

Additionally, we can initialize an iterator variable for our vector of moves and monitors:
{\tt \begin{snugshade*}
\begin{lstlisting}
mvi = 0
mni = 0
\end{lstlisting}
\end{snugshade*}}

Finally, we create a helper variable that specifies the number of states that the morphological character has.
{\tt \begin{snugshade*}
\begin{lstlisting}
NUM_STATES = 2
\end{lstlisting}
\end{snugshade*}}
Using this variable we can easily change our script to use a different morphological character with a different number of states.
We will also use this variable in our second example on hidden-state speciation and extinction model. 



\subsection{Specifying the model}

The basic idea behind the model in this example is that speciation and extinction rates are dependent on a binary character \citep{Maddison2007}.


\subsubsection{Priors on rates}
We start by specifying prior distributions on the diversification rates.
We will assume here an identical prior distribution on the speciation and extinction rate.
Furthermore, we will use a normal distribution as the prior distribution on the log of the speciation and extinction rate.
Hence, we will use a mean of $\frac{\ln(\frac{\text{\#Taxa}}{2})}{\text{tree-age}}$ which is the expected net-diversification rate.
{\tt \begin{snugshade*}
\begin{lstlisting}
rate_mean <- ln( ln(367.0/2.0) / observed_phylogeny.rootAge() )
rate_sd <- 2.0
\end{lstlisting}
\end{snugshade*}}
Now we can specify our character-specific specification and extinction rate parameters.
As we just said before, we are going to use normal distributions for the prior on the log-speciation and log-extinction rate.
Here we will use a \cl{for}-loop to specify speciation and extinction parameters for each character, \EG two in a binary state case.
{\tt \begin{snugshade*}
\begin{lstlisting}
for (i in 1:NUM_STATES) {
    
     ### Create a lognormal distributed variable for the diversification rate
    log_speciation[i] ~ dnNormal(mean=rate_mean,sd=rate_sd) 
    speciation[i] := exp( log_speciation[i] )
    moves[++mvi] = mvSlide(log_speciation[i],delta=0.20,tune=true,weight=3.0)

    ### Create a lognormal distributed variable for the turnover rate
    log_extinction[i] ~ dnNormal(mean=rate_mean,sd=rate_sd) 
    extinction[i] := exp( log_extinction[i] )
    moves[++mvi] = mvSlide(log_extinction[i],delta=0.20,tune=true,weight=3.0)

}
\end{lstlisting}
\end{snugshade*}}

{\tt \begin{snugshade*}
\begin{lstlisting}
Create a rate-matrix for the relative-rate of change between categories.
# Each transition rate between observed states are drawn
# from an exponential distribution with a mean of 10
# character state transitions over the tree. 
rate_pr := observed_phylogeny.treeLength() / 10
rate_12 ~ dnExp(rate_pr)
rate_21 ~ dnExp(rate_pr)

moves[++mvi] = mvScale( rate_12, weight=2 )
moves[++mvi] = mvScale( rate_21, weight=2 )

rate_matrix := fnFreeBinary( [rate_12, rate_21 ], rescaled=false)
\end{lstlisting}
\end{snugshade*}}


\subsubsection{Prior on the root state}
Create a constant variable with the prior probabilities of each rate category at the root.
{\tt \begin{snugshade*}
\begin{lstlisting}
rate_category_prior ~ dnDirichlet( rep(1,NUM_STATES) )
moves[++mvi] = mvDirichletSimplex(rate_category_prior,tune=true,weight=2)
\end{lstlisting}
\end{snugshade*}}

\subsubsection{Incomplete Taxon Sampling}

We know that we have sampled 233 out of 367 living primate species. 
To account for this we can set the sampling parameter as a constant node with a value of 233/367
{\tt \begin{snugshade*}
\begin{lstlisting}
rho <- observed_phylogeny.ntips()/367
\end{lstlisting}
\end{snugshade*}}


\subsubsection{Root age}

The birth-death process requires a parameter for the root age.
In this exercise we use a fix tree and thus we know the age of the tree.
Hence, we can get the value for the root from the \citet{MagnusonFord2012} tree.
{\tt \begin{snugshade*}
\begin{lstlisting}
root <- observed_phylogeny.rootAge()
\end{lstlisting}
\end{snugshade*}}

\subsubsection{The time tree}

Now we have all of the parameters we need to specify the full episodic birth-death model. 
We initialize the stochastic node representing the time tree.
{\tt \begin{snugshade*}
\begin{lstlisting}
timetree ~ dnCDBDP( rootAge           = root,
                    speciationRates   = speciation,
                    extinctionRates   = extinction, 
                    Q                 = rate_matrix,
                    pi                = rate_category_prior,
                    delta             = 1.0,
                    rho               = rho,
                    condition         = "survival",
                    taxa              = taxa )
\end{lstlisting}
\end{snugshade*}}
And then we attach data to it.
{\tt \begin{snugshade*}
\begin{lstlisting}
timetree.clamp( observed_phylogeny )
timetree.clampCharData( data )
\end{lstlisting}
\end{snugshade*}}

Finally, we create a workspace object of our whole model using the \cl{model()} function. 
{\tt \begin{snugshade*}
\begin{lstlisting}
mymodel = model(rate_matrix)
\end{lstlisting}
\end{snugshade*}}

The \cl{model()} function traversed all of the connections and found all of the nodes we specified. 


\subsection{Running an MCMC analysis}

\subsubsection{Specifying Monitors}

For our MCMC analysis, we set up a vector of \emph{monitors} to record the states of our Markov chain. 
{\tt \begin{snugshade*}
\begin{lstlisting}
monitors[++mni] = mnFile(filename="output/primates_BiSSE_"+DATASET+".trees", printgen=1, timetree)
monitors[++mni] = mnModel(filename="output/primates_BiSSE_"+DATASET+".log", printgen=1)
monitors[++mni] = mnJointConditionalAncestralState(tree=timetree, cdbdp=timetree, type="Standard", printgen=1, withTips=true, withStartStates=false, filename="output/anc_states_primates_BiSSE_"+DATASET+".log")
monitors[++mni] = mnScreen(printgen=10, rate_12, rate_21, speciation, extinction)
\end{lstlisting}
\end{snugshade*}}

\subsubsection{Initializing and Running the MCMC Simulation}

With a fully specified model, a set of monitors, and a set of moves, we can now set up the MCMC algorithm that will sample parameter values in proportion to their posterior probability. The \cl{mcmc()} function will create our MCMC object:
{\tt \begin{snugshade*}
\begin{lstlisting}
mymcmc = mcmc(mymodel, monitors, moves)
\end{lstlisting}
\end{snugshade*}}

First, we will run a pre-burnin to tune the moves and to obtain starting values from the posterior distribution.
{\tt \begin{snugshade*}
\begin{lstlisting}
mymcmc.burnin(generations=5000,tuningInterval=200)
\end{lstlisting}
\end{snugshade*}}

Now, run the MCMC:
{\tt \begin{snugshade*}
\begin{lstlisting}
mymcmc.run(generations=20000)
\end{lstlisting}
\end{snugshade*}}


\subsubsection{Summarizing ancestral states}

{\tt \begin{snugshade*}
\begin{lstlisting}
anc_states = readAncestralStateTrace("output/anc_states_primates_BiSSE_solitariness.log")
anc_tree = ancestralStateTree(tree=observed_phylogeny, ancestral_state_trace_vector=anc_states, include_start_states=false, file="output/anc_states_primates_BiSSE_solitariness_results.tree", burnin=0, summary_statistic="MAP", site=0)
\end{lstlisting}
\end{snugshade*}}


\subsubsection{Plotting ancestral states}

\definecolor{shadecolor}{RGB}{200,200,200}
{\tt \begin{snugshade*}
\begin{lstlisting}
library(RevGadgets)

tree_file = "output/anc_states_primates_BiSSE_results.tree"

plot_ancestral_states(tree_file, summary_statistic="MAP",
                      tip_label_size=0,
                      xlim_visible=NULL,
                      node_label_size=0,
                      show_posterior_legend=TRUE,
                      node_size_range=c(2, 6),
                      alpha=0.75)

output_file = "RevBayes_Anc_States_BiSSE.pdf"
ggsave(output_file, width = 11, height = 9)
\end{lstlisting}
\end{snugshade*}}
\definecolor{shadecolor}{RGB}{183,207,237}

\subsubsection{Plotting diversification rates}


\definecolor{shadecolor}{RGB}{200,200,200}
{\tt \begin{snugshade*}
\begin{lstlisting}
library(coda)
library(ggplot2)
source("scripts/multiplot.R")

data <- read.table("output/primates_BiSSE.log",header=TRUE)

dat_ext  <- data.frame(dens = c(data$extinction.1, data$extinction.2), Type = rep(c("1", "2"), each = length(data$extinction.1)))
dat_spec <- data.frame(dens = c(data$speciation.1, data$speciation.2), Type = rep(c("1", "2"), each = length(data$extinction.1)))
dat_div  <- data.frame(dens = c(data$speciation.1-data$extinction.1, data$speciation.2-data$extinction.2), Type = rep(c("1", "2"), each = length(data$extinction.1)))
dat_rel  <- data.frame(dens = c(data$extinction.1/data$speciation.1, data$extinction.2/data$speciation.2), Type = rep(c("1", "2"), each = length(data$extinction.1)))

pdf("RevBayes_BiSSE_Results.pdf")

p1 <- ggplot(dat_spec, aes(x = dens, fill = Type)) + labs(title = "Speciation", x="Rate", y="Posterior Density") + geom_density(alpha = 0.5)
p2 <- ggplot(dat_ext, aes(x = dens, fill = Type)) + labs(title = "Extinction", x="Rate", y="Posterior Density") + geom_density(alpha = 0.5)
p3 <- ggplot(dat_div, aes(x = dens, fill = Type)) + labs(title = "Net-Diversification", x="Rate", y="Posterior Density") + geom_density(alpha = 0.5)
p4 <- ggplot(dat_rel, aes(x = dens, fill = Type)) + labs(title = "Relative Extinction", x="Rate", y="Posterior Density") + geom_density(alpha = 0.5)

multiplot(p1, p2, p3, p4)
dev.off()
\end{lstlisting}
\end{snugshade*}}



\definecolor{shadecolor}{RGB}{183,207,237}

%\impmark{The \Rev file for performing this analysis: \href{https://github.com/revbayes/revbayes_tutorial/raw/master/RB_DiversificationRateEpisodic_Tutorial/RevBayes_scripts/mcmc_EBD.Rev}{\cl{mcmc\_EBD.Rev}}.}
%\impmark{An \R file for plotting the output: \href{https://github.com/revbayes/revbayes_tutorial/raw/master/RB_DiversificationRateEpisodic_Tutorial/RevBayes_scripts/Plot_EBD_RevBayes.R}{\cl{Plot\_EBD\_RevBayes.R}}.}


%\subsection{Exercise}

%\begin{itemize}
%\item Run an MCMC simulation to estimate the posterior distribution of the speciation rate and extinction rate.
%\item Visualize the rate through time using \R.
%\item Do you see evidence for rate decreases or increases? What is the general trend?
%\item Run the analysis using a different number of intervals, \EG 5 or 50. How do the rates change?
%\item Modify the model by specifying a prior on the log-diversification and log-turnover rate and then estimate the diversification rates through time. Do you see any differences in the estimates? 
%\end{itemize}




\bibliographystyle{sysbio}
\bibliography{\GlobalResourcePath refs}
