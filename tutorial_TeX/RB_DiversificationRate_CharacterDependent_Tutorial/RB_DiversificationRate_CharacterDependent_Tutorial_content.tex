\section{Estimating Speciation \& Extinction Rates Through Time}

\subsection{Outline}

This tutorial describes how to specify a character-dependent branching-process models in \RevBayes;
a birth-death process where diversification rates are dependent on the state of a discrete character \citep{Maddison2007,Fitzjohn2012}.
The probabilistic graphical model is given for this tutorial.
Finally, you will estimate speciation and extinction rates through-time using Markov chain Monte Carlo (MCMC).


\subsection{Requirements}
We assume that you have read and hopefully completed the following tutorials:
\begin{itemize}
\item RB\_Getting\_Started
\item RB\_Basics\_Tutorial
\item RB\_BayesFactor\_Tutorial
\item RB\_BasicDiversificationRate\_Tutorial
\end{itemize}
Note that the RB\_Basics\_Tutorial introduces the basic syntax of \Rev but does not cover any phylogenetic models.
You may skip the RB\_Basics\_Tutorial if you have some familiarity with \R.
The RB\_BayesFactor\_Tutorial introduced Bayesian model selection by means of Bayes factors, which can be skipped by readers familiar with Bayesian model selection.
We tried to keep this tutorial very basic and introduce all the language concepts and theory on the way.
You may only need the RB\_Basics\_Tutorial for a more in-depth discussion of concepts in \Rev.


%%%%%%%%
%%   Data   %%
%%%%%%%%
\section{Data and files}

We provide the data file(s) which we will use in this tutorial.
You may want to use your own data instead.
In the \cl{data} folder, you will find the following files
\begin{itemize}
\item \cl{primates\_springer.tre}: Dated primates phylogeny including 369 out of 450 species from .
\end{itemize}


\impmark{Open the tree \cl{data/primates\_springer.tre} in FigTree.}


\bigskip
\section{Character-dependent diversification rates}\label{sec:CDBDP}


\subsection{Read the tree}

Begin by reading in the observed tree. 

{\tt \begin{snugshade*}
\begin{lstlisting}
T <- readTrees("data/Primates_tree.nex")[1]
data <- readDiscreteCharacterData("data/Primates_morph.nex")
\end{lstlisting}
\end{snugshade*}}
On this example we are only going to use the first character of the matrix.
{\tt \begin{snugshade*}
\begin{lstlisting}
data.excludeAll()
data.includeCharacter(1)
\end{lstlisting}
\end{snugshade*}}

From this tree, we can get some helpful variables:
{\tt \begin{snugshade*}
\begin{lstlisting}
taxa <- T.taxa()
\end{lstlisting}
\end{snugshade*}}

Additionally, we can initialize an iterator variable for our vector of moves:
{\tt \begin{snugshade*}
\begin{lstlisting}
mvi = 0
\end{lstlisting}
\end{snugshade*}}

Finally, we create a helper variable that specifies the number of intervals.
{\tt \begin{snugshade*}
\begin{lstlisting}
NUM_STATES = 2
\end{lstlisting}
\end{snugshade*}}
Using this variable we can easily change our script to break-up time into many or few intervals.



\subsection{Specifying the model}

The basic idea behind the model in this example is that speciation and extinction rates are dependent on a binary character.


\subsubsection{Priors on rates}
We start by specifying prior distributions on the rates.

{\tt \begin{snugshade*}
\begin{lstlisting}
diversification_mean <- ln( ln(450.0/2.0) / T.rootAge() )
turnover_mean <- ln( ln(450.0/2.0) / T.rootAge() )
diversification_sd <- 0.587405*2
turnover_sd <- 0.587405*2
\end{lstlisting}
\end{snugshade*}}


{\tt \begin{snugshade*}
\begin{lstlisting}
for (i in 1:NUM_STATES) {

    ### Create a lognormal distributed variable for the diversification rate
    diversification[i] ~ dnLognormal(mean=diversification_mean,sd=diversification_sd) 
    moves[++mvi] = mvScale(diversification[i],lambda=1.0,tune=true,weight=3.0)

    ### Create a lognormal distributed variable for the turnover rate
    turnover[i] ~ dnLognormal(mean=turnover_mean,sd=turnover_sd) 
    moves[++mvi] = mvScale(turnover[i],lambda=1.0,tune=true,weight=3.0)


    ### Transform the parameters
    speciation[i] := ln( diversification[i] + turnover[i] )
    extinction[i] := ln( turnover[i] ) 

}
\end{lstlisting}
\end{snugshade*}}

Create a rate-matrix for the relative-rate of change between categories.
Here we simply use equal rates, that is,  a Jukes-Cantor rate-matrix with \cl{NUM\_STATES} states.
{\tt \begin{snugshade*}
\begin{lstlisting}
rate_matrix <- fnJC( NUM_STATES )
\end{lstlisting}
\end{snugshade*}}

Create a constant variable with the prior probabilities of each rate category at the root.
{\tt \begin{snugshade*}
\begin{lstlisting}
rate_category_prior <- simplex( rep(1,NUM_STATES) )
\end{lstlisting}
\end{snugshade*}}

Create a exponential distributed variable for the rate of rate-shift events
{\tt \begin{snugshade*}
\begin{lstlisting}
event_rate ~ dnExponential(1.0)
moves[++mvi] = mvScale(event_rate,lambda=1,tune=true,weight=5)
\end{lstlisting}
\end{snugshade*}}



\subsubsection{Incomplete Taxon Sampling}

We know that we have sampled 369 out of 450 living primate species. 
To account for this we can set the sampling parameter as a constant node with a value of 369/450
{\tt \begin{snugshade*}
\begin{lstlisting}
rho <- T.ntips()/450
\end{lstlisting}
\end{snugshade*}}


\subsubsection{Root age}

The birth-death process requires a parameter for the root age.
In this exercise we use a fix tree and thus we know the age of the tree.
Hence, we can get the value for the root from the \citet{Springer2012} tree.
{\tt \begin{snugshade*}
\begin{lstlisting}
root_time <- T.rootAge()
\end{lstlisting}
\end{snugshade*}}

\subsubsection{The time tree}

Now we have all of the parameters we need to specify the full episodic birth-death model. 
We initialize the stochastic node representing the time tree.
{\tt \begin{snugshade*}
\begin{lstlisting}
timetree ~ dnCDBDP( rootAge                   = root,
                    hiddenSpeciationRates     = [ 0.0 ],
                    observedSpeciationRates   = speciation,
                    hiddenExtinctionRates     = [ 0.0 ],
                    observedExtinctionRates   = extinction, 
                    Q                         = rate_matrix,
                    delta                     = event_rate,
                    pi                        = rate_category_prior,
                    rho                       = rho,
                    condition                 = "survival",
                    taxa                      = taxa )
\end{lstlisting}
\end{snugshade*}}
And then we attach data to it.
{\tt \begin{snugshade*}
\begin{lstlisting}
timetree.clamp( T )
timetree.clampCharData( data )
\end{lstlisting}
\end{snugshade*}}

Finally, we create a workspace object of our whole model using the \cl{model()} function. 
{\tt \begin{snugshade*}
\begin{lstlisting}
mymodel = model(rho)
\end{lstlisting}
\end{snugshade*}}

The \cl{model()} function traversed all of the connections and found all of the nodes we specified. 


\subsection{Running an MCMC analysis}

\subsubsection{Specifying Monitors}

For our MCMC analysis, we set up a vector of \emph{monitors} to record the states of our Markov chain. 
{\tt \begin{snugshade*}
\begin{lstlisting}
monitors[1] = mnModel(filename="output/primates_CBBD.log",printgen=10, separator = TAB)
monitors[2] = mnScreen(printgen=10, event_rate)
\end{lstlisting}
\end{snugshade*}}

\subsubsection{Initializing and Running the MCMC Simulation}

With a fully specified model, a set of monitors, and a set of moves, we can now set up the MCMC algorithm that will sample parameter values in proportion to their posterior probability. The \cl{mcmc()} function will create our MCMC object:
{\tt \begin{snugshade*}
\begin{lstlisting}
mymcmc = mcmc(mymodel, monitors, moves)
\end{lstlisting}
\end{snugshade*}}

First, we will run a pre-burnin to tune the moves and to obtain starting values from the posterior distribution.
{\tt \begin{snugshade*}
\begin{lstlisting}
mymcmc.burnin(generations=10000,tuningInterval=200)
\end{lstlisting}
\end{snugshade*}}


Now, run the MCMC:
{\tt \begin{snugshade*}
\begin{lstlisting}
mymcmc.run(generations=50000)
\end{lstlisting}
\end{snugshade*}}


%\impmark{The \Rev file for performing this analysis: \href{https://github.com/revbayes/revbayes_tutorial/raw/master/RB_DiversificationRateEpisodic_Tutorial/RevBayes_scripts/mcmc_EBD.Rev}{\cl{mcmc\_EBD.Rev}}.}
%\impmark{An \R file for plotting the output: \href{https://github.com/revbayes/revbayes_tutorial/raw/master/RB_DiversificationRateEpisodic_Tutorial/RevBayes_scripts/Plot_EBD_RevBayes.R}{\cl{Plot\_EBD\_RevBayes.R}}.}


%\subsection{Exercise}

%\begin{itemize}
%\item Run an MCMC simulation to estimate the posterior distribution of the speciation rate and extinction rate.
%\item Visualize the rate through time using \R.
%\item Do you see evidence for rate decreases or increases? What is the general trend?
%\item Run the analysis using a different number of intervals, \EG 5 or 50. How do the rates change?
%\item Modify the model by specifying a prior on the log-diversification and log-turnover rate and then estimate the diversification rates through time. Do you see any differences in the estimates? 
%\end{itemize}




\bibliographystyle{sysbio}
\bibliography{\GlobalResourcePath refs}
