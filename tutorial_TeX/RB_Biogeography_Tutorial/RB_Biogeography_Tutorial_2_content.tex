\section{Epoch models and ancestral range reconstruction}

\subsection{Ancestral range reconstruction}


From the previous section, we created an ancestral state monitor by
\begin{snugshade}
\begin{lstlisting}
mn[3] = mnJointConditionalAncestralState(tree=psi, ctmc=m, filename=out_fn+".states.txt", type="NaturalNumbers", printgen=10, withTips=true, withStartStates=true)
\end{lstlisting}
\end{snugshade}

The joint-conditional ancestral state monitor samples the joint distribution of ancestral states every {\tt printgen} iterations over the entire tree, conditioning on the observed tip states.
Due to cladogenesis, the state before speciation and the states inherited after speciation may differ, which we monitor setting {\tt withStartStates=true}.
For convenience we record the tip states using {\tt withTips=true} though these are known through the input data.

The resulting file, {\tt bg\_1.states.txt}, appears as

\begin{framed}
\begin{lstlisting}
Iteration	start_0	end_0	start_1	end_1	start_2	end_2	start_3	end_3	...
0		2	2	2	4	2	2	2	6	...
10		2	2	4	4	2	2	2	6	...
20		2	2	2	4	2	2	6	6	...
...
\end{lstlisting}
\end{framed}

where columns give the integer-valued range corresponding to either the start or end of a branch leading to a particular node, and each row corresponds to a single sample drawn from the joint distribution of ancestral states.
For example, the lineage leading to the node indexed 1 starts in state 2 (Africa) and ends in state 4 (Eurasia) at iteration 20.
In Section \ref{sec:bayarea}, we will look at how stochastic mappings---the completely realized biogeographic history---may be analysed for data-augmented models.
If you wish to write your own scripts to analyse the ancestral range reconstructions, know that each node's index is recorded when writing a tree's Newick string to file so states may be mapped on to the tree.

\subsection{Data exploration with Phylowood}

To interpret the biogeographical history of primates, we will generate a Phylowood (\url{http://mlandis.github.io/phylowood}) animation.
First, we'll create variables for the relevant files

\begin{snugshade}
\begin{lstlisting}
state_fn = fp + "output/bg_1.states.txt"
atlas_fn = fp + "data/earth3.still.atlas.txt"
phw_fn = fp + "output/bg_1.phw.txt"
\end{lstlisting}
\end{snugshade}

then create the animation file

\begin{snugshade}
\begin{lstlisting}
convertToPhylowood(treefile=tree_fn, geofile=atlas_fn, statefile=state_fn, outfile=phw_fn, burnin=0., type="range")
\end{lstlisting}
\end{snugshade}

This file summarizes the MCMC output from a RevBayes biogeographical analysis as a Nexus-formatted file, which is used by Phylowood to generate interactive animations to explore biogeographic reconstructions.

\noindent \\ \impmark Open \texttt{http://mlandis.github.io/phylowood}.

\noindent \\ \impmark Drag and drop \texttt{./output/bg\_1.phw.txt} into the text field.

\begin{figure}[H]
\centering
\includegraphics[width=4in]{figures/bg_1_mrca}
\caption{Phylowood frame showing posterior ancestral range of root node.}
\end{figure}

\noindent \\ \impmark Click the Play button to view the animation. \\

Notice the primate MRCA range is very widespread with six areas. This is

There are three control panels to help you filter data: the media panel, the map panel, and the phylogeny panel.
The media buttons correspond to Beginning, Slow/Rewind, Play, Stop, Fast Forward, Ending (from left to right).
The animation will play the timeframe corresponding to the slider.

\noindent \\ \impmark Drag the slider to the right (the present).

\begin{figure}[H]
\centering
\includegraphics[width=4in]{figures/bg_1_tips}
\caption{Phylowood frame showing distribution of extant taxon ranges.}
\end{figure}

\noindent \\ \impmark Pan and zoom around the map.\\

Marker colors correspond to the phylogenetic lineages in the phylogeny panel.
Markers are split into slices and (loosely) sorted phylogenetically, so nearby slices are generally closely related.
At divergence events, a marker's radius is proportional to the marginal posterior probability the node was present in the area at that time.
Between divergence events, marker's radius is simply an interpolation of the values at the two endpoints.
Some information about geological constraints and cladogenic events is lost.

\noindent \\ \impmark Mouseover an area to learn which lineage it belongs to and its presence probability. \\

Since it's difficult to see how specific clades evolve with so many taxa, Phylowood offers two ways to filter taxa from the animation.
We call the set of a lineage, all its ancestral lineages towards the root, and all descendant lineages a phylogenetic heritage.
The root's heritage is the entire clade.
A leaf node's heritage is a path from the tip to the root.

\noindent \\ \impmark Mouseover a lineage to temporarily highlight the lineage's heritage. Remove the mouseover to remove the highlight effect. \\

The highlight effect is temporary and quickly allows you to single out lineages of interest during animation.
Phylowood also offers a masking effect that persists until an unmask command is issued.

\noindent \\ \impmark Double-click the white root branch to mask the root node's heritage (all lineages). Single click a lineage to unmask that lineage's heritage. \\

\begin{figure}[H]
\centering
\includegraphics[width=4in]{figures/bg_1_loris}
\caption{Phylowood frame highlighting the ancestral range for the MRCA extant lorises.}
\end{figure}

Now that the masking effects are in place, you're free to interact with other map components.
In addition, the area of marker sizes is only distributed among unmasked lineages.

Visit \texttt{https://github.com/mlandis/phylowood/wiki} to learn more about Phylowood.

\subsection{Epoch models}

(under construction)

