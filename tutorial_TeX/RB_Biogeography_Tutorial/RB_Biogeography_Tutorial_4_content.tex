
%%%%%%%%%%%%%%%%
%%%%%%%%%%%%%%%%

\section{Biogeographic dating using DEC} \label{sec:bg_phylo}

This analysis will jointly estimate phylogeny and biogeography.
One benefit is that the biogeographic analysis will intrinsically accommodate phylogenetic uncertainty, both in terms of topology and branch lengths.
Another is that paleogeographic evidence has the potential provide information about the geological timing of speciation events in the phylogeny \citep{Landis2016}.
Finally, biogeographic data may lend support to certain phylogenetic relationships that have poor resolution otherwise.

As mentioned in Section \ref{sec:bg_simple}, Hawaiian silverswords are nested in a larger group of plants, the tarweeds.
Fossil pollen evidence indicates that tarweeds diversified during a period of aridification from 15--5 Ma in the western regions of North America \citep{Baldwin1991}.
Although the oldest Hawaiian island that silverswords inhabit is Kauai, it is possible that silverswords first colonized older islands in the Emperor Island chain that predate the formation of Kauai at about 5.1 Ma.

This makes traditional node-based biogeographic calibrations challenging, because it would require a strong assumption about when and how many times the oldest silversword lineages colonized Kauai.
Did silverswords colonize Kauai once directly from the California coast? Or did the colonize the younger islands multiple times from older islands in the chain? And did the event occur immediately after Kauai surfaced or much later? Because we cannot observe the timing and nature of this event directly, this process-based biogeographic dating approach does so through probabilistic inference.

\begin{figure}[!ht]
\centering
\includegraphics[width=\textwidth]{figures/fig_biogeo_dating.png}
\caption{Cartoon of biogeographic transition probabilities as functions of geological time, and how that relates to speciation times. (a) Areas split, dispersal before split, positive probability; (b) Areas split, dispersal after split, zero probability; (c) Areas merge, dispersal after merge, positive probability; (d) Areas merge, dispersal before merge, zero probabilty. Figure from \citet{Landis2016}.}
\label{fig:biogeo_dating_cartoon}
\end{figure}

To address these issues, it is necessary that we transition from our simpler 4-area model to a richer 6-area model (see Figure \ref{fig:hawaii_areas}).
The mainland area (Z) is necessary to force the silversword and tarweed clade to originate apart from the islands.
The area corresponding to the older island chain (R) is necessary because we do not know {\it a priori} whether silverswords colonized the modern islands directly from Z, or first colonized R and only later dispersed into the younger islands any number of times.
This permits the silversword origin time to precede the formation of Kauai when the dispersal rate is large.

Additionally, we will add three tarweed taxa to our dataset, increasing the total number of taxa to 38.
We'll use internal transcribed spacer (ITS) to estimate the phylogeny, which is a 600 bp non-coding locus that is historically important for plant systematics.
Because the locus is relatively short, it will also leave us with a fair amount of phylogenetic uncertainty in branch length and topology estimates.
However, because we're estimating phylogeny and biogeography, it will be correctly incorporated into our ancestral range estimates.

Note: like all dating methods, including node calibration methods, tip dating methods, and fossilized birth death dating methods, process-based biogeographic dating estimates are prior sensitive and dataset dependent.
Applying this model to alternative data sets should be done with care!

Much of this tutorial will be similar to the previous sections, except we are adding a birth-death process and a molecular substitution process to the model graph.

\subsection*{Analysis}

Create the necessary input/output variables.

\begin{snugshade}
\begin{lstlisting}
range_fn = "data/n6/silversword.n6.range.nex"
mol_fn = "data/n6/silversword.mol.nex"
tree_fn = "data/n6/silversword.tre"
out_fn = "output/test_epoch_phy"
geo_fn = "data/n6/hawaii.n6"
times_fn = geo_fn + ".times.txt"
dist_fn = geo_fn + ".distances.txt"
\end{lstlisting}
\end{snugshade}

Add the analysis helper variables

\begin{snugshade}
\begin{lstlisting}
mvi = 1
mni = 1
n_gen = 1e5    # more parameters, longer run!
\end{lstlisting}
\end{snugshade}


Read in the molecular alignment

\begin{snugshade}
\begin{lstlisting}
dat_mol = readDiscreteCharacterData(mol_fn)
\end{lstlisting}
\end{snugshade}


Impose limits to the maximum range size, both to prohibit widespread species ranges and to improve the computational efficiency of the method.

First, get the number of areas
\begin{snugshade}
\begin{lstlisting}
dat_range_01 = readDiscreteCharacterData(range_fn)
n_areas <- dat_range_01.nchar()
\end{lstlisting}
\end{snugshade}

Suppose we wanted to forbid ranges from being three or more areas in size.
The total number of ranges is $\sum_{k=0}^m {{n}\choose{k}}$ where $n$ is the total number of areas, $m$ is the maximum number of permissible areas, and ${{n}\choose{k}}$ is the number of ways to sample $k$ unordered areas from a pool of $n$ areas.

\begin{snugshade}
\begin{lstlisting}
max_areas <- 2
n_states <- 0
for (k in 0:max_areas) n_states += choose(n_areas, k)
\end{lstlisting}
\end{snugshade}

Then format the dataset for the reduced state space

\begin{snugshade}
\begin{lstlisting}
dat_range_n = formatDiscreteCharacterData(dat_range_01, "DEC", n_states)
\end{lstlisting}
\end{snugshade}


% read in some geography data

Read in the list of minimum and maximum ages of island formation

\begin{snugshade}
\begin{lstlisting}
time_bounds <- readDataDelimitedFile(file=times_fn, delimiter=" ")
n_epochs <- time_bounds.size()
\end{lstlisting}
\end{snugshade}

Read in a vector of matrices that describe the connectivity between areas over time.
Note, there is one connectivity matrix per epoch, ordered from oldest to youngest.

\begin{snugshade}
\begin{lstlisting}
for (i in 1:n_epochs) {
  epoch_fn[i] = geo_fn + ".connectivity." + i + ".txt"
  connectivity[i] <- readDataDelimitedFile(file=epoch_fn[i], delimiter=" ")
}
\end{lstlisting}
\end{snugshade}

Read in the matrix of distances between all pairs of areas (km). For simplicity, we will assume that distances remain constant over time, even though they certainly vary.

\begin{snugshade}
\begin{lstlisting}
distances <- readDataDelimitedFile(file=dist_fn, delimiter=" ")
\end{lstlisting}
\end{snugshade}


For this analysis, all starting trees should have a non-zero probability.
However, in general, it is possible that certain combinations of phylogeny, biogeography, and paleogeography have zero-valued likelihoods should the epoch model introduce reducible rate matrix structures.
In addition, the initial MCMC state must have a non-zero probability for it to work properly.
For MCMC to operate correctly, it must be initialized with a state with non-zero probability.
See the supplemental of \citet{Buerki2011} for an explanation.

\begin{snugshade}
\begin{lstlisting}
tree_init = readTrees(tree_fn)[1]
\end{lstlisting}
\end{snugshade}


Store some basic information about the taxon set, the number of taxa, and the number of branches in the tree

\begin{snugshade}
\begin{lstlisting}
taxa = tree_init.taxa()
n_taxa = taxa.size()
n_branches = 2 * n_taxa - 2
\end{lstlisting}
\end{snugshade}

\subsubsection{The tree model}

In this exercise we will also be estimating the phylogeny (topology and branch lengths), meaning our tree will be a stochastic node with a prior distribution.
For this, we'll use a constant rate birth-death process.

Assign root age with a maximum age of 15Ma to reflect the fossil pollen record for Californian tarweeds [cite].

\begin{snugshade}
\begin{lstlisting}
root_age ~ dnUniform(0, 15)
moves[mvi++] = mvScale(root_age, weight=2)
\end{lstlisting}
\end{snugshade}

Assign the proportion of sampled taxa (we have a non-uniform sampling scheme, but this should suffice).
\begin{snugshade}
\begin{lstlisting}
rho <- 35/50
\end{lstlisting}
\end{snugshade}

Assign the birth and death priors.
It is important to note that the birth and death priors induce a root age distribution through the birth-death process.
These priors generate a relatively uniform root age distribution between 2.5--15 Ma in the absence of data (i.e. running MCMC with the {\tt underPrior=true} option).
\begin{snugshade}
\begin{lstlisting}
birth ~ dnExp(1)
moves[mvi++] = mvScale(birth)
death ~ dnExp(1)
moves[mvi++] = mvScale(death)
\end{lstlisting}
\end{snugshade}

Instantiate a tree variable generated by a birth-death process
\begin{snugshade}
\begin{lstlisting}
tree ~ dnBDP(lambda=birth, mu=death, rho=rho, rootAge=root_age, taxa=taxa)
\end{lstlisting}
\end{snugshade}


Add topology and branch length moves
\begin{snugshade}
\begin{lstlisting}
moves[mvi++] = mvNNI(tree, weight=n_branches/2)
moves[mvi++] = mvFNPR(tree, weight=n_branches/8)
moves[mvi++] = mvNodeTimeSlideUniform(tree, weight=n_branches/2)
\end{lstlisting}
\end{snugshade}

Provide a starting tree (improves mixing, not essential)

\begin{snugshade}
\begin{lstlisting}
tree.setValue(tree_init)
root_age.setValue(tree_init.rootAge())
\end{lstlisting}
\end{snugshade}


\subsubsection{The molecular model}

In addition, to inform our branch lengths (in relative time units) and our topology, we will specify a simple HKY+$\Gamma4$+UCLN model of molecular substitution.


First specify a base rate for the molecular clock. This prior is uniform over orders of magnitude, between $10^-6$ and $10^3$
\begin{snugshade}
\begin{lstlisting}
log10_rate_mol ~ dnUniform(-6, 3)
log10_rate_mol.setValue(-1)
moves[mvi++] = mvSlide(log10_rate_mol, weight=5, delta=0.2)
rate_mol := 10^log10_rate_mol
\end{lstlisting}
\end{snugshade}

Assign log-normal relaxed clock rate multipliers to each branch in the tree. These priors have a mean of 1 so each branch prefers a strict clock model in the absence of data.
\begin{snugshade}
\begin{lstlisting}
branch_sd <- 1.0
branch_mean <- 0.0 - 0.5 * branch_sd^2
for (i in 1:n_branches) {
    branch_rate_multiplier[i] ~ dnLognormal(mean=branch_mean, sd=branch_sd)
    moves[mvi++] = mvScale(branch_rate_multiplier[i])
    branch_rates[i] := rate_mol * branch_rate_multiplier[i]
}
\end{lstlisting}
\end{snugshade}

Now we'll create an HKY rate matrix. First the transition-transversion rate ratio (with prior with mean=1)

\begin{snugshade}
\begin{lstlisting}
kappa ~ dnGamma(2,2)
moves[mvi++] = mvScale(kappa)
\end{lstlisting}
\end{snugshade}

the base frequencies over A, C, G, and T

\begin{snugshade}
\begin{lstlisting}
bf ~ dnDirichlet([1,1,1,1])
moves[mvi++] = mvSimplexElementScale(bf, alpha=10, weight=2)
\end{lstlisting}
\end{snugshade}

then using the base frequencies and TsTv rate ratio to build the matrix

\begin{snugshade}
\begin{lstlisting}
Q_mol := fnHKY(kappa, bf)
\end{lstlisting}
\end{snugshade}

Next, we'll create a $+\Gamma4$ across site rate variation model.
This requires a parameter to control how much site rate heterogeneity there is.

\begin{snugshade}
\begin{lstlisting}
alpha ~ dnUniform(0,50)
moves[mvi++] = mvScale(alpha)
\end{lstlisting}
\end{snugshade}

and a discretized Gamma distribution with 4 categories

\begin{snugshade}
\begin{lstlisting}
site_rates := fnDiscretizeGamma(alpha, alpha, 4)
\end{lstlisting}
\end{snugshade}

When {\tt alpha} is large, then the Gamma distribution centers its density around the rate multiplier of 1, meaning that all sites evolve at similar rates.
When {\tt alpha} is small, the Gamma distribution presents more site rate heterogeneity.

Finally, we'll create our molecular model of substitution

\begin{snugshade}
\begin{lstlisting}
m_mol ~ dnPhyloCTMC(Q=Q_mol, tree=tree, branchRates=branch_rates, siteRates=site_rates, type="DNA", nSites=dat_mol.nchar())
\end{lstlisting}
\end{snugshade}

and attach the ETS dataset

\begin{snugshade}
\begin{lstlisting}
m_mol.clamp(dat_mol)
\end{lstlisting}
\end{snugshade}


\subsubsection{The biogeographic model}
% distances between areas

The biogeographic model is identical to that described in Section \ref{sec:bg_epoch}.

First, create biogeographic rate parameter.

\begin{snugshade}
\begin{lstlisting}
log10_rate_bg ~ dnUniform(-4,2)
log10_rate_bg.setValue(-2)
rate_bg := 10^log10_rate_bg
moves[mvi++] = mvSlide(log10_rate_bg, weight=4)
\end{lstlisting}
\end{snugshade}


The relative dispersal rate is fixed to 1
\begin{snugshade}
\begin{lstlisting}
dispersal_rate <- abs(1)
\end{lstlisting}
\end{snugshade}


the distance scale parameter

\begin{snugshade}
\begin{lstlisting}
distance_scale ~ dnUnif(0,20)
distance_scale.setValue(0.001)
moves[mvi++] = mvScale(distance_scale, weight=3)
\end{lstlisting}
\end{snugshade}


Now we can assign rates that are functions of distance between all pairs of areas

%dr[i][j][k] := (a * exp( -b * distances[j][k] ))
\begin{snugshade}
\begin{lstlisting}
for (i in 1:n_epochs) {
  for (j in 1:n_areas) {
    for (k in 1:n_areas) {
     dr[i][j][k] <- abs(0)
     if (connectivity[i][j][k] > 0) {
       dr[i][j][k] := dispersal_rate * exp(-distance_scale * distances[j][k])
     }
    }
  }
}
\end{lstlisting}
\end{snugshade}


% extirpation penalized ranges
% ... they can exist, but not persist

It is unlikely that widespread ranges persist across disjunct areas for long periods of time.
Extirpation is more likely to occur in fragmented ranges than well-connected ranges, where peripheral populations are continuously reinforced from the center.

\begin{snugshade}
\begin{lstlisting}
log_sd <- 0.5
log_mean <- ln(1) - 0.5*log_sd^2
extirpation_rate ~ dnLognormal(mean=log_mean, sd=log_sd)
moves[mvi++] = mvScale(extirpation_rate, weight=2)

for (i in 1:n_epochs) {
  for (j in 1:n_areas) {
    for (k in 1:n_areas) {
      er[i][j][k] <- abs(0.0) 
    }
    er[i][j][j] := extirpation_rate
  }
}

\end{lstlisting}
\end{snugshade}


Build a rate matrix for each time interval
\begin{snugshade}
\begin{lstlisting}
for (i in 1:n_epochs) {
  Q_DEC[i] := fnDECRateMatrix(dispersalRates=dr[i],
                          extirpationRates=er[i],
                          maxRangeSize=max_areas)
}
\end{lstlisting}
\end{snugshade}



% uncertainty in paleogeographic events

Treat epoch times as random variables. The present is always the present.


\begin{snugshade}
\begin{lstlisting}
for (i in 1:n_epochs) {
  time_max[i] <- time_bounds[i][1]
  time_min[i] <- time_bounds[i][2]
  if (i != n_epochs) {
    epoch_times[i] ~ dnUniform(time_min[i], time_max[i])
    moves[mvi++] = mvSlide(epoch_times[i], delta=(time_bounds[i][1]-time_bounds[i][2])/2)
  } else {
    epoch_times[i] <- 0.0
  }
}
\end{lstlisting}
\end{snugshade}


% epoch model for anagenetic change


Create the epoch rate generator object
\begin{snugshade}
\begin{lstlisting}
Q_DEC_epoch := fnEpoch(Q=Q_DEC, times=epoch_times, rates=rep(1, n_epochs))
\end{lstlisting}
\end{snugshade}

% clado event probs

Here, we treat the probability of different types of cladogenetic events as a random variable to be estimate.

\begin{snugshade}
\begin{lstlisting}
clado_event_types <- [ "s", "a" ]
p_sympatry ~ dnUniform(0,1)
p_allopatry := abs(1.0 - p_sympatry)
moves[mvi++] = mvSlide(p_sympatry, delta=0.1, weight=2)
clado_event_probs := simplex(p_sympatry, p_allopatry)
P_DEC := fnDECCladoProbs(eventProbs=clado_event_probs,
                         eventTypes=clado_event_types,
                         numCharacters=n_areas,
                         maxRangeSize=max_areas)
\end{lstlisting}
\end{snugshade}

For this dataset, we assume cladogenetic probabilities are constant with respect to geological time.



\begin{snugshade}
\begin{lstlisting}
rf_DEC <- rep(0, n_states)
rf_DEC[n_areas+1] <- 1  # Mainland (Z) is the only possible starting state
rf_DEC <- simplex(rf_DEC)
\end{lstlisting}
\end{snugshade}


Create the phylogenetic model
\begin{snugshade}
\begin{lstlisting}
m_bg ~ dnPhyloCTMCClado(tree=tree,
                           Q=Q_DEC_epoch,
                           cladoProbs=P_DEC,
                           branchRates=rate_bg,
                           rootFrequencies=rf_DEC,
                           type="NaturalNumbers",
                           nSites=1)        
\end{lstlisting}
\end{snugshade}


Attach the dataset
\begin{snugshade}
\begin{lstlisting}
m_bg.clamp(dat_range_n)
\end{lstlisting}
\end{snugshade}


Add a deterministic node to monitor the crown age of the silversword radiation

\begin{snugshade}
\begin{lstlisting}
ingroup_clade <- clade("Wilkesia_hobdyi",
                       "Dubautia_reticulata",
                       "Dubautia_microcephala",
                       "Argyroxiphium_caliginis")

ingroup_age := tmrca(tree, ingroup_clade)
\end{lstlisting}
\end{snugshade}


Add add nodes that report the posterior probability that the clade originates before or after a given island.
When the first argument in of the {\tt ifelse} function returns {\tt true}, the node has value {\tt 1} and {\tt 0} otherwise.
Thus, the mean of this variable gives the posterior probability that the inequality is satisfied.

\begin{snugshade}
\begin{lstlisting}
for (i in 1:n_epochs) {
  ingroup_older_island[i] := ifelse(ingroup_age > epoch_times[i], 1, 0)
}
\end{lstlisting}
\end{snugshade}



\begin{snugshade}
\begin{lstlisting}
monitors[mni++] = mnScreen(printgen=100, ingroup_age)
monitors[mni++] = mnModel(file=out_fn+".model.log", printgen=100)
monitors[mni++] = mnFile(tree, filename=out_fn+".tre", printgen=100)
monitors[mni++] = mnJointConditionalAncestralState(tree=tree,
                                                       ctmc=m_bg,
                                                       type="NaturalNumbers",
                                                       withTips=true,
                                                       withStartStates=true,
                                                       filename=out_fn+".states.log",
                                                       printgen=100)
\end{lstlisting}
\end{snugshade}



Compose the model object.
Because {\tt ingroup\_older\_island} does not contribute to the model likelihood, it must be manually introduced to the model object.

\begin{snugshade}
\begin{lstlisting}
mymodel = model(m_bg, ingroup_older_island)
\end{lstlisting}
\end{snugshade}


\begin{snugshade}
\begin{lstlisting}
mymcmc = mcmc(mymodel, moves, monitors)
mymcmc.run(n_gen)
\end{lstlisting}
\end{snugshade}


\subsection*{Results}

{\it Example results are provided as {\tt output\_example/epoch\_phy.*} and {\tt output\_example/simple\_phy.*} }

To understand the influence of the epoch model on ancestral range and divergence time estimation, it is important to run addition analyses with alternative settings.
Scripts to jointly estimate molecular evolution, historical biogeographic, and phylogenetic parameters are available as {\tt scripts/run\_simple\_phy.Rev} and {\tt scripts/run\_epoch\_phy.Rev}.
The ``epoch'' analysis is identical to the analysis just described.
The ``simple'' analysis is similar to the ``epoch'' analysis, except it substitutes the paleogeography-aware model of range evolution (see Section \ref{sec:bg_epoch}) for a paleogeography-naive model (see Section \ref{sec:bg_simple}).

\begin{figure}[!ht]
\centering
\includegraphics[width=0.45\textwidth]{figures/fig_simple_phy_RevGadgets_ase.pdf}\includegraphics[width=0.45\textwidth]{figures/fig_epoch_phy_RevGadgets_ase.pdf} 
\caption{Joint estimate of phylogeny and biogeography. The left panel ignores paleogeography while the right panel conditions on it.}
\label{fig:epoch_phy}
\end{figure}

We see that simple analysis (Fig \ref{fig:epoch_phy}, Left) estimates the ancestral range at the root of the clade as Maui + Mainland (MZ).
This is unrealistic, both because of the extreme distance between those areas, but also the simple analysis estimates the root age to be 10.3 (HPD95\% 4.6, 15.0) Ma, well before Maui originated.

The epoch analysis (Fig \ref{fig:epoch_phy}, Right) produces more sensible ancestral range estimates, with Kauai being colonized first, and younger islands only being colonized as they become available.
When comparing the results to the earlier fixed-phylogeny epoch results \ref{fig:epoch_RevGadgets_ase}, we recover a greater role for cladogenesis for the younger speciation events.
These two analyses only differ in terms of whether the phylogeny is fixed or estimated, so it is likely a result of phylogenetic error in the fixed tree.

\begin{figure}[!h]
\centering
\includegraphics[width=0.49\textwidth]{figures/fig_simple_ages.pdf} \includegraphics[width=0.49\textwidth]{figures/fig_epoch_ages.pdf} 

\caption{Plot of trace for island ages and the origin time of living silverswords. The left panel ignores paleogeography, allowing silverswords to originate well before the formation of Kauai ({\tt epoch\_times[1]}). The right panel conditions of paleogeography, which prefers a silversword crown age that follows the formation of Kauai.}
\label{fig:epoch_ages}
\end{figure}

In Tracer, one can look at the sampled posterior of island ages in comparison the origination time of crown silverswords (Fig \label{fig:epoch_ages}).
The left panel shows the simple analysis, where crown silverswords often originate before the formation of Kauai.
The right panel shows that crown silverswords probably originated before the formation of Maui, but after the formation of Kauai.

\begin{table}[!h]
\centering
\begin{tabular}{c|cccc}
Model       & $P(a_s>a_K)$ & $P(a_s>a_O)$ & $P(a_s>a_M)$ & $P(a_s>a_H)$ \\ \hline
simple & 0.72 & 0.94 & 0.99 & 1.00 \\
epoch & 0.02 & 0.26 & 0.84 & 0.99 \\
\end{tabular}
\caption{Posterior probability that the age of crown silverswords ($a_s$) is older than the origination times of K, O, M, and H ($a_K, a_O, a_M, a_H$, respectively). The ``simple'' model (Left) ignores paleogeography while the ``epoch'' model (Right) conditions on it.}
\label{tab:epoch_ages}
\end{table}

By tabulating the results of the deterministic variable {\tt ingroup\_older\_island}, we measure the posterior probability that crown silverswords originated before or after each particular epoch in the model (Table \ref{tab:epoch_ages}).


\newpage
